% Resume inspired from learnings from: a) Preet Mam from ITM communication skills training, b) Codechef Resume Meme, c) Kaggle Resume Youtube Video
% ~TODO~ DONE: Make sure the resume respects above three's rules
%
% Last updated on: 29th July 2025 by Vikas Prasad
%-------------------------
% Resume in Latex
% Author : Sourabh Bajaj
% License : MIT
%------------------------

\documentclass[letterpaper,11pt]{article}

\usepackage{latexsym}
\usepackage[empty]{fullpage}
\usepackage{titlesec}
\usepackage{marvosym}
\usepackage[usenames,dvipsnames]{color}
\usepackage{verbatim}
\usepackage{enumitem}
\usepackage[hidelinks]{hyperref}
\usepackage{fancyhdr}
\usepackage[english]{babel}

\pagestyle{fancy}
\fancyhf{} % clear all header and footer fields
\fancyfoot{}
\renewcommand{\headrulewidth}{0pt}
\renewcommand{\footrulewidth}{0pt}

% Adjust margins
\addtolength{\oddsidemargin}{-0.5in}
\addtolength{\evensidemargin}{-0.5in}
\addtolength{\textwidth}{1in}
\addtolength{\topmargin}{-.5in}
\addtolength{\textheight}{1.0in}

\urlstyle{same}

\raggedbottom
\raggedright
\setlength{\tabcolsep}{0in}

% Sections formatting
\titleformat{\section}{
  \vspace{-4pt}\scshape\raggedright\large
}{}{0em}{}[\color{black}\titlerule \vspace{-5pt}]

%-------------------------
% Custom commands
\newcommand{\resumeItem}[2]{
  \item\small{
    \textbf{#1}{: #2 \vspace{-2pt}}
  }
}

\newcommand{\resumeItemMinimal}[1]{
  \item\small{
    {#1 \vspace{-2pt}}
  }
}

\newcommand{\resumeSubheading}[2]{
  \vspace{-1pt}\item
    \begin{tabular*}{0.97\textwidth}[t]{l@{\extracolsep{\fill}}r}
      \textbf{#1} & #2 \\
    \end{tabular*}
}

\newcommand{\resumeSubSubheading}[2]{
    \begin{tabular*}{0.97\textwidth}[t]{l@{\extracolsep{\fill}}r}
      \textit{\small#1} & \textit{\small #2} \\
    \end{tabular*}\vspace{-5pt}
}

\newcommand{\resumeSubItem}[2]{\resumeItem{#1}{#2}\vspace{-4pt}}

\renewcommand{\labelitemii}{$\circ$}

\newcommand{\resumeSubHeadingListStart}{\begin{itemize}[leftmargin=*]}
\newcommand{\resumeSubHeadingListEnd}{\end{itemize}}
\newcommand{\resumeItemListStart}{\begin{itemize}}
\newcommand{\resumeItemListEnd}{\end{itemize}\vspace{-5pt}}


%-------------------------------------------
%%%%%%  CV STARTS HERE  %%%%%%%%%%%%%%%%%%%%%%%%%%%%


\begin{document}

%----------HEADING-----------------
\begin{tabular*}{\textwidth}{l@{\extracolsep{\fill}}r}
  \textbf{\Large Vikas Prasad} & \href{mailto:vikasprasad.prasad@gmail.com}{vikasprasad.prasad@gmail.com}\\
  LinkedIn: \href{https://www.linkedin.com/in/vikas--prasad/}{\color{blue}vikas--prasad}; GitHub: \href{https://github.com/viiicky}{\color{blue}viiicky}; Stack Overflow: \href{http://stackoverflow.com/users/1781024/vikas-prasad}{\color{blue}1781024} & +91 7566564744 \\
\end{tabular*}

%-----------EXPERIENCE-----------------
\section{Experience}
  \resumeSubHeadingListStart

    \resumeSubheading
      {Swym}{Remote}
      \resumeSubSubheading
        {Engineering Manager}{Aug 2024 - Present}
        \resumeItemListStart
          \resumeItem{Shopify Markets \& Infrastructure}
            {Led implementation of complex \textbf{Shopify Markets} feature in Wishlist Plus application, onboarding real merchants. Led infrastructure initiatives including \textbf{Kubernetes} migration and \textbf{MongoDB} 6.0 to 8.0 upgrade. Worked on cost-cutting initiatives by optimizing \textbf{MongoDB} data management through TTLs, indices, and pruning older data, saving \textbf{\$1,200/month}.}
          \resumeItem{Organizational Development}
            {Created and implemented \textbf{Career Path} documentation for Engineering organization. Pushed for \textbf{public conversation by default} model and knowledge sharing culture. Established regular \textbf{1:1s}, manager surveys, and Engineering Leaders syncups.}
        \resumeItemListEnd

    \resumeSubheading
      {Fyle (acquired by Sage)}{Bengaluru, India \& Remote}
      \resumeSubSubheading
        {Engineering Manager 2}{Jan 2022 - Aug 2023}
        \resumeItemListStart
          \resumeItem{Leadership \& Team Development}
            {Received executive coaching on \textbf{situational leadership}, \textbf{influencing}, and \textbf{executive presence}. Successfully promoted \textbf{all 6} team members and resolved conflicts. Achieved \textbf{98.4\%} bug resolution rate in fire brigade team.}
          \resumeItem{Data Extraction Optimization}
            {Boosted DE accuracy significantly: category in paper receipts from \textbf{14.84\%} to \textbf{67.49\%}, date in digital receipts from \textbf{39.96\%} to \textbf{73.65\%}. Reduced team costs by \textbf{90\%} (\textbf{USD 8600 to USD 950/month}) and eliminated legacy services. \href{https://aws.amazon.com/blogs/startups/how-amazon-textract-helped-fyle-boost-data-extraction-accuracy/}{\textbf{https://aws.amazon.com/blogs/startups/how-amazon-textract-helped-fyle-boost-data-extraction-accuracy/}}}
            % We improved the accuracy so much that we got rid of transcribers. We had 3 humans hired that were working on it. Once we made our systems confident enough, we got rid of them.
            % And, how did we improve the accuracy - we moved from custom models to Textract, and then on top of it we wrote our custom layers to further optimize the accuracy, these custom layers could be regex, or other custom logics.
            % And as part of working on this we had to frequently talk to AWS folks, follow up with them over the email thread, as this feature was yet to be released to everyone, and we were one of the early customers for them.

            % https://stories.fylehq.com/p/how-did-we-increase-data-extraction
            % like policy this was again a 5 year old service needing overhaul long due
            % I joined the DE team a year ago as a noob and today I run the show. The things that I take care of now include maintaining the service, building the test infrastructure, analyzing all old services + old test infra deeply and merging all 5+ years of work to a single new service in just a month’s time, sounds awesome right? Of course, it is AWESOME!
            % Our test infrastructure helped us get the clean and slick stats presented before, having the infra helped us push things confidently to production at insane speeds.

            % https://stories.fylehq.com/p/implementing-good-engineering-practices
            % For tasks having a big impact on application stability like Python upgrades, linting & formatting the whole codebase at once, refactoring codebase, and critical library upgrades having code coverage of project above acceptable percentage e.g. greater than 90% becomes very helpful. Having code covered gives mental peace to developers and improves confidence while working on big changes. It is somewhat like having insurance, we pay by adding tests, and whenever do big changes these tests pay us back!
            % Easy POCs/experiments: If good enough code coverage then doing POCs/experiments like library upgrades, python upgrades, and OS distribution changes can be done very easily by validating its compatibility by runnings tests.

            % Expense rules
            % https://stories.fylehq.com/p/building-expense-rules-a-journey
          \resumeItem{Cost Reduction and Legacy Cleanup}
            {Reduced Data Extraction team costs by \textbf{90\%}, \textbf{from USD 8600 to USD 950 per month}, and completely eliminated legacy DE services and repos.}
        \resumeItemListEnd
        % Also worked on fying expenses via SMS
        % Also worked on Expense rules

      \resumeSubSubheading
        {Engineering Manager 1}{Jan 2021 - Dec 2021}
        \resumeItemListStart
          \resumeItem{Leadership Training \& Team Management}
            {Completed "Become 10x Manager in 4 weeks!" cohort learning tough conversations, \textbf{1:1s}, and actionable feedback.}
            % - Onboarding a new engineer (Madhav) who had mostly worked on frontend-tech to data-extraction team successfully and smoothly. Thanks to the guidance from Siva on how to make the life of dev easier. In the beginning I was unsure how would someone with so less experience will be able to pull of the responsibility of handling whole data-extraction, but nevertheless I kept at it and tried to make Madhav successful in owning DE.
            % - There was a point where Madhav was at quite a low point wrt to DE tests initiative because of the pace of the progress. It took a lot of patience and active listening for me to gather his worries and insecurities and then guide him through pushing out of it successfully.
            % - Data extraction: Pushed the team to make DEv2 live. It lost steam a bit in the mid, but then we managed, and it ended really well. We had been receiving good feedback from different teams throughout. We were also able to boost up the accuracy for most of the fields.
            % Apart from the initiative, also coached Madhav on how to better manage tasks, especially bigger initiatives. This showed very good results wrt Madhav having confidence while working on complex projects, as he could see the smaller milestones being achieved and checkmarked slowly.
            % - Pushed Kirti as deploy folk: Kirti showed interest in deploy related activities. I think this is one of those examples, where I was able to take care of my report in a very timely fashion. The moment he talked about it in our 1x1 - I talked to Gokul, Kartikey and Kirti individually to assess what each of them individually thinks about this idea and Kirti’s capability. Post this, I was able to outline where the gaps are, and what all things were potentially stopping Kirti to reach there. I laid out the plan, got green from all the stakeholders on the plan, shared the common plan with all the stakeholders along with Kirti so that there is visibility, and it all happened right as per the plan.
            % Over here, of course, the actual work was done by Kirti, what I feel I did better here was, listen to the report’s need actively and then timely taking an action on it along with having a concrete tangible plan.
            % - Pushed Irfan as a fire brigade member: Irfan has now been in the fire brigade team for more than 2 consecutive months now. This has not only given made his desire of knowing other parts of the codebase possible but has also helped the fire brigade team as anything related to policy(and in some cases, other workflows too), he is able to handle quickly. There is a scope development here possible though - when he is the most experienced person the fire brigade call, he needs to own the show(feedback already given)
            % - Madhav interview stint: After being reported from Vaishnavi + Kartikey about Madhav's approach to interview, I immediately talked to Madhav and very patiently guided him on how to make an interview-review successful and how to see a candidate. Why I am putting this in the best work question? Because over here, I applied something that I very recently learnt from Siva. Siva recently shared detailed textual feedback for Abhishek Kumar while he was working on it. It takes time to write a detailed explanation out, but I found it quite valuable and I thought going forward I would imply this learning, and when this situation arose, I actually took the time out and whatever I talked to him about, actually wrote down and shared with him to develop this skill that I newly observed from Siva.
            \resumeItem{Processes \& Documentation}
            {Created \textbf{Manager's Readme}: \href{https://billowy-cardamom-348.notion.site/Vikas-Prasad-Fyle-4c5aa13f228142f493cf5d19021f79b8}{\textbf{https://billowy-cardamom-348.notion.site/Vikas-Prasad-Fyle-4c5aa13f228142f493cf5d19021f79b8}}. Ran \textbf{PIP} for 1, career reviews for \textbf{9} folks. Led policy migration from Python 2 to 3, established high test coverage standards, and created new experts for policy codebase.}
            % - Irfan’s PIP: This was a task where you need to be very disciplined. You cannot miss to send the daily status email at any cost, and you need to review the work on the daily basis. This was the first time I ran the PIP for someone, and it was a good experience overall.
            % - Appraisals: It was the first time I did review and appraisals. For ~9 folks. Appraisal conversations are definitely one of the toughest conversations I have had, and though it was quite exhausting, I like that it was a challenging task and not something easy. I also am thankful to Siva, for guiding me through it. Without his support, I could not imagine, how would this have gone /\
            % - Was able to finally set decent engineering practises(with Siva’s support), quality documentation and testing setup(Postman, integration, and backtest) for data-extraction. Onboarded Madhav with the help of these same docs without any issue.
            % - Rolled out a couple of offer letters to interns. Doesn't really fit in the "best work" question per se. But I am adding it here, because since I did it for the first time, it felt quite good :)
            % - 1x1: 1x1 was one of the action items that was captured from my survey. I got better at it and was able to do more meaningful 1x1s this quarter. There is still possibility for me for more development here as I recently learned quite a few things about 1x1 from the lead cohort, which I want to try and apply here. But still, it’s better than where it was earlier.
            % - Manager’s readme: After constant pushes from Siva and Abhishek, I was finally able to push out the [manager’s readme](https://www.notion.so/4c5aa13f228142f493cf5d19021f79b8?pvs=21) for myself. It was taken by all the team members very very positively. Reports are now more aware of my quirks and behaviour and know how to get around them to keep things moving forward. Even outside Fyle(example in the lead cohort), I have gotten very good feedback about the readme, which also made me feel better.
            % The only thing left here is to try it out with a new team member and see how does it play out there. 

            % - policy migration from python 2 to python 3: This was a gigantic task, and I wrote the whole migration, and I was writing code after a while so all of it was perfect. Only thing, because of my emergency holidays, I was not there went it was being made live, but I am proud of the fact that the team was able to continue and push it through even when I was not around - just like a fault tolerant system. Of course, this was possible because of Kartikey and Siva handling the team in my absence.
            % - policy tests: Again, I was not around when this was being made live, and unlike the above task, I did not write bigger chunk of it. I laid the foundation and the initial discussions with Kirti and Irfan, specially making them understand the bigger picture for this initiative were interesting. For this particular item, more than me, the credit goes to Kartikey(and Siva) and the team. I am glad that we were able to set up such a high number for test coverage, which basically now works as a gold-standard for other teams.
            % - policy experts: Again, here I was not very confident how it would turn out, but it all worked out at the end. This is about creating new experts for the policy codebase, and transferring the load from Kartikey to Kirti + Irfan. Both the complex initiative and the team work actually worked out and we are able to successfully create two relatively lesser experience folks as the expert of a such a complex codebase in a matter of just a quarter. This particular item, I am really very amazed and proud about.
            % - Policy tests blog: Was finally able to push the team and make the blog happen. As this was the first blog from Kirti and Irfan, I had to do a lot of ground-level work. While it required a lot of my time, I am also sure that both of them are in a better position to write out future blogs. In fact, for Kirti I do have this idea(which I am yet to push out to him) of writing a blog on how he ended up being a member of the deploy team - right from he showing interest, doing courses in his free time all the way to gaining the confidence of other deploy team members. I want him to successfully finish this quarter wrt his deploy responsibilities and then talk about it.
            % Policy tests: In this blog, we share the story of how we added tests support to a 5-year-old micro-service with 4K+ lines of code and 600+ commits. https://stories.fylehq.com/p/from-zero-to-hero-the-policy-tests
            % "Can we leave it on staging for a week or two, just in case" - we asked nervously.
            % Before policy tests, there was always this feeling of uncertainty - fear of breaking something. So we would test it rigorously with the help of other devs and leave it on testing environments because we feared the regression. As a result, the code changes didn't take as much time as the testing itself took, leading to delayed policy releases and bug fixes.
            % 96% code coverage
            % As a result, the policy code went from the most blamed code (for bugs) to one of the cleanest codes (pre-Diwali cleaning) in Fyle codebase.

          % Others:
          % - There were a total of 14 items on me for Q1. Out of which 12 were made GA within time. A couple of these were ad-hoc too, but handled with ease.

        % See full self assessment here:
        % - https://www.notion.so/Quarterly-career-review-75f8f1910b4d42fdbcc00326b5b4be79?pvs=4#ac486b895ebc4e6c8c1c7d3000699bb9
        % - https://www.notion.so/Quarterly-career-review-75f8f1910b4d42fdbcc00326b5b4be79?pvs=4#f1ee3d2d34cd4e2bbcd7b91ad89b58e9
        % - https://www.notion.so/Quarterly-career-review-75f8f1910b4d42fdbcc00326b5b4be79?pvs=4#c7a52a96af564191bbb6157ac68375fd
        \resumeItemListEnd

      \resumeSubSubheading
        {Senior Member of Technical Staff 2}{Jan 2020 - Dec 2020}
        \resumeItemListStart
            % Sync up skills: One of the goals I set for myself in Q4 was to improve my sync-up skills.
            % It was Q3 when I realized the priority of this problem and set up a plan for myself for Q4.
            % I made sure that both the operational and non-operational sync ups with all the 10 folks I worked in Q4 were being finished and documented.
            % Till before Q4, I would typically reschedule, or avoid non-operational 1:1s.
            % I made sure to take down this issue head-on by having documented 1:1 with direct reportees twice a month, and skip level once a month.
            % What made me realize that this^ was a problem:
            % There are two things I could recall:
            %   a) When Adithya was transitioning, and I had to take over policies and data-extraction teams, and I did the initial call with each of the team members from these teams.
            %   It was Prashant, who mentioned how it’s been quite long that he had non-operational with Adithya, and he had quite a lot to share, like what he was promised, what is his current state, what were his frustrations etc.
            %   It was then, when it striked me how I also miss my non-operationals quite often, and how this can lead my team members to a similar state as Prashant.
            %   b) When I was on a call with you, again during around Adithya’s transition, and when I mentioned, that if there is one kind of meeting from my calendars, which I skip or reschedule most, it is non-operational 1x1, and then you mentioned, how this is exact opposite, and it is this meeting that should be prioritised the most and you recommended to do it every week with direct reportees and every other week with skip levels.
            %   Although, I was convinced by the importance of it by now, I was not confident about the frequency proposed, given that there were a total of 10 folks, I had been working with directly/indirectly last quarter, and thus I adapted it to every two weeks with direct and every 1 month with skip level. This itself was a decent aim for me given my then state.  
        
        \resumeItem{Infrastructure \& Guidelines}
            {Created high-leverage fyle-stats infrastructure, wrote \textbf{Haskell} code for \textbf{postgrest}, and developed exhaustive JAVA PR guidelines. Led customer interviews and implemented Monday commitments \& Friday celebrations practices.}
            % Customer interviews: There were a couple of initiatives where I lead the customer interviews. This went very well. Set up a process and template within the team for this.
            % Prepared for interviews in advance. Did detailed analysis of interview sessions. Documented everything.
            % And, above all, taught and coached all of these to the team leads working with me, which they are now finding very useful when leading interviews on their own.
            % As we were doing this for the first time, I first needed to set an example with one or two initiatives so that later others can learn from them, and that is exactly how it went.
            
            % Team practices and processes: In Q4, since we moved to the new way of doing things,
            % and, there were 10 folks I was working with directly or indirectly, by the end of Q3 and beginning of Q4,
            % I set this internal practice of Monday commitments and Friday celebrations meetings. This is the idea that I got from reading the book Radical Focus.
            % I thought of giving it a shot, and it was received pretty well.
            % Every single person whom I talked with, mentioned that Monday planning session is helping them in different ways.
            % For Friday sessions there was a couple of feedback, which now is being overridden by Aditya's show-off initiative.
            % Similarly, I got this idea and got it setup where we implemented a quick dynamic AMP email to collect feedback from the customers.
            % My aim here was twofolds - not just give customers the chance to share fresh feedback, but also any praises directly coming to a public slack channel straight from the customer will give immense gratification feeling to the dev working on it.
            % Sample from #amp_report_approval: https://fylein.slack.com/archives/C01BKQKRJJF/p1605002130025100
            % Sample from #fyle_expenses_through_amp_email: https://fylein.slack.com/archives/C01CWFGRWSU/p1610698347003100
            % On the same lines, I got the equivalent idea setup inside the webapp for #make_budgets_more_powerful in a generic fashion,
            % as a result of which not only initiatives like #simplify_expense_policy_form or #bring_more_powerful_workflows started using the same component,
            % but initiatives from others teams like Vashnavi’s also reused the same component to collect fresh feedback from customers :)

            % Github Issue: https://github.com/PostgREST/postgrest/issues/1573
            % Haskell Code: https://github.com/viiicky/postgrest/commits/master

            % stats-infra: Wrote `fyle-stats`. I like this because of two reasons:
            % 1) this is high leverage task. I wrote the infra, and now the complete app is able to show stats on each of the list views or any other place, by writing minimal code/configurations and following a generic contract. This enables people to enable stats support for a given list view within minutes.
            % 2) As part of this, in order to unblock me, I had to fork `postgrest` and write some code. Interesting part is, this codebase was in Haskell. Had never ever thought that I would ever write anything in Haskell. Had to break my head over understanding pure functional programming, and then write the patch. It was a very interesting challenge.
            % Link to Github issue and the Haskell code that we wrote, mentioned above.
            % On top of all this, there is not even a single bug reported in stats-infra from the day it went live. ��

            % JAVA PR guidelines doc: Went through the past 100 PRs, collected the recurring comments people have been receiving.
            % Then talked to the top reviewers, collected the frequent comments they have been giving.
            % Added the frequent comments I have been giving.
            % Compiled all of this together, got reviewed by all the top reviewers, and then finally put the exhaustive doc out followed by a session to the Engineering team.
            % Post this, many such guidelines doc followed, by different members of Engineering team covering different languages.

          % See full self assessment here: https://www.notion.so/Quarterly-career-review-75f8f1910b4d42fdbcc00326b5b4be79?pvs=4#61c7408ff9df41f8a03f813ab01b1b0a
        \resumeItemListEnd

      \resumeSubSubheading
        {Member of Technical Staff}{Jan 2018 - Jan 2020}
        \resumeItemListStart
          \resumeItem{MIS-Analytics \& Jobs Infrastructure}
            {Led module development, created large \textbf{Postgres} materialized view (180+ columns, 30+ joins), improved refresh time by \textbf{100x+}. Built end-to-end background job processing framework using \textbf{APScheduler}, \textbf{RabbitMQ}, \textbf{Celery}. \href{https://stories.fylehq.com/p/engineering-guide-to-food-trucks}{\textbf{https://stories.fylehq.com/p/engineering-guide-to-food-trucks}}}
            % Also, persistent filters
            % Also, business case and functional requirements: custom reporting

            %---------detailed description-----------
            % Developed GD service from scratch. Created big fat Postgres materialised view(180+ columns; 30+ joins) for mis and analytics purpose around expenses.
            % Later, switched to custom refresh mechanism(on normalised tables) bringing refresh time down from 55 mins to 30 sec.
            % Adapted the same solution for other core objects like reports, payments.
            % Designed analytics support for various kinds of charts. Designed and led many other subsequent services/features/improvements.
            % Developed v1: GD service from scratch for MIS
            % Developed v2: Introduced custom incremental/full refresh mechanism bringing data refresh time from 55 minutes to 30 seconds i.e. 100x+ improvement
            % Designed v3: Analytics support
            % Led(either no design by me or just requirements by me) other similar services: mis-reports, mis-payments. Also subsequent new features like new charts, frequently used filters on Analytics page, full refresh overhaul using time based approach.
            % (2 people team other than me, both dev)
            % Designed and lead other various subsequent features: Attachments urls in company expenses export

            %---------gold-digger related facts----------
            % fact table for gold-digger: transactions
            % fat table: gold_digger_schema.transaction_collections
            % no. of transactions = 2171988
            % size of transactions table = 4665 MB
            % size of gold_digger_schema.transaction_collections = 8311 MB
            % number of columns in gold_digger_schema.transaction_collections = 183
            % number of joins for gold_digger_schema.transaction_collections table definition = 34
            % helpful blog to get the whole picture(we are not the author): https://stories.fylehq.com/p/analytics-pipeline-using-incremental
            % Common approach to analytics is to have a set of normalised tables and load data from core /fast-moving (source) tables to these separate views/table definitions (destination) -  optimised for running complex analytical queries

            % End-to-end means, not only development, but pre and post dev stuff also, like: business requirements, functional requirements, engineering design, dev+documentation+testing+demo+maintenance etc.
            % Learnt really a good loads of things from this project. Got many myths busted :)

            % Other items:
            % - Extended existing service to support any new integration in a generic fashion. Added Netsuite, SunGL and GEFU integration using this generic support.
            % Added support for \textbf{QBO} locations \& classes. Even worked on JavaScript. Constantly \textbf{maintained} and \textbf{supported} one top QBO using customer.
            % developed the above thing
            % Also did: UI work here
            % constantly maintained and supported the UK client FAE
            % - {Led \& developed various features/services, enforced \textbf{good practices}, \& contributed to engineering \textbf{hiring} (9 to 90 employees in 2 years). Also involved in \textbf{production deployments} \& \textbf{customer success} for bugs.}
            % Also enforced \textbf{good practices} and \textbf{initiatives} like using editorconfig, BugsNPizza etc.
            % Was also actively involved in engineering \textbf{hiring}(seen the company grow from 9 to 90 members in around 2 years), and briefly involved in handling \textbf{production deployments} and \textbf{customer success} on bugs front.}
            % - developed end to end Default Cost Center Support for both web app and mobile
            % - developed end to end Default Cost Center Support bulk edition;
            % - developed end to end Allowed per diem features
            % - developed and documented Bulk invite API for Hero's Adrenaline software (temporary solution till TPA support was being created)
            % - wrote Hero specific MIS scripts
            % - developed Auto generate unique sequential payment numbers at org level
            % - developed JwtPrincipal support for including roles, scopes, added_by etc.
            % - led defaults-handler
          \resumeItem{Notifications}
          {Led notifications module, AMP reply to comments, email service \& delayed emails.}
            % (4 people team other than me, 2 dev, 1 designer, 1 dev intern)
        \resumeItemListEnd
% What is Fyle: Intelligent Expense Management Software. a new standard for expense management that provides one click expense tracking for employees and unmatched control for enterprises.
% Tech stack at Fyle: main service in Java with Dropwizard framework. Almost all others services in Python with Flask framework. Every now and then people keep trying different tech like Ruby, Django etc. etc.
% Some other tech stack: jdbi, Postgres, rabbitMQ, Angular, ionic framework. Google Guice for dependency injection in Java and Liquibase for migrations in Java, Alembic for migrations in Python.
% Joined as the 9th employee on 2018 New Year, and as of writing we have 90 employees - in a span of almost 2 years, and still hiring.

    \resumeSubheading
      {Infosys Limited}{Mysuru \& Bengaluru, India}
      \resumeSubSubheading
        {Systems Engineer}{Dec 2015 - Dec 2017}
        \resumeItemListStart
          \resumeItem{Connected Car Platform}
            {Developed Notification and Vehicle Locator services using \textbf{Spring}, \textbf{Java}, \textbf{mongoDB}, and \textbf{AWS}.}
  % What: Developed Notification and Vehicle Locator service from scratch. Worked on Spring Boot, Spring Data, Java, JUnit, Maven, git, mongoDB. Got introduced to mockito, REST-assured, Java 8 features: Instant, Lambda, Supplier, Optional, reflections, Design Patterns: Factory, State, Singleton, Builder, Jackson, Batch Processing, Spring Scheduler, Spring State Machine, Spring JMS Listener, Spring Integration, Azure Service Bus, Qpid, sonarQube, twilio, SendGrid, amazon web services: Amazon SES, Amazon SNS, RequestBin, ngrok, JFrog, Jenkins, VSTS, semaphoreci, kibana. Got introduced to processes like: Agile, TDD, Pair Programming.
  % Author tag on spring-library :) https://github.com/spring-projects/spring-integration/commit/74b59ef8d3d157c7d19e7937314812fc5a9c15e6

          \resumeItem{Digital Oil Field}
            {Worked on UI for managing oil fields digitally, adding functionality for custom plots and separator views using \textbf{AngularJS}, \textbf{Node.js}, \textbf{MSSQL}.}
  % What: UI App Full Stack Developer, Mssql, AngularJS, Node.js � DOF is an application which collects data through various sensors placed in oil field and then after filtering those data, provides interactive dashboard to the user where user can:
  % - monitor the activities of oil wells, various equipment like separators, pump etc.
  % - close alarms raised by any equipment.
  % - plot various graphs between different parameters to see the production details
  % - forecast productions, etc.
  %
  % Phase 1 of the application was already developed. Data was being taken from Pi Historian. Middleware was written in C# & MSSQL was used as the database. UI app was written using AngularJS JavaScript framework. And Node.js was used as the backend of the UI app. Worked on the UI app for Phase 2. Added functionality for saving custom plots. Added views for separators along with various other fixes. Improved AngularJS, Bootstrap & Node.js knowledge. Got introduced to D3, NVD3, MSSQL & Leaflet
  % Where: ICETS, Infosys, BLR
        \resumeItemListEnd

    \resumeSubheading
      {Free and Open Source Software for Education (FOSSEE), IIT Bombay}{Remote}
      \resumeSubSubheading
        {Intern}{April 2015 - May 2015}
        \resumeItemListStart
          \resumeItem{Textbook Companion Programme}
            {\href{http://tbc-python.fossee.in/book-details/502/}{\textbf{http://tbc-python.fossee.in/book-details/502/}}}
        \resumeItemListEnd
% % What: Textbook Companion Programme by IIT Bombay on Internet
% % What did I do:
% - Contributed to creating reference material for the book \textbf{Data Structures} \& \textbf{Algorithms Analysis}. Received honorarium.
% - Submitted Python code for all the solved examples of the book Sams Teach Yourself Data Structures & Algorithms Analysis in 24 Hours; FOSSEE (Free and Open Source Software for Education); Contributed in creating a repository of reference material. Received honorarium. Strengthened Python, Object Oriented and Data Structures Algorithms Analysis skills. Complete project on above link.
% % When: April 2015 - May 2015
  \resumeSubHeadingListEnd

%-----------EDUCATION-----------------
\section{Education}
  \resumeSubHeadingListStart
  
    \resumeSubheading
      {Institute of Technology and Management}{Gwalior, India}
      \resumeSubSubheading
      {Bachelor of Engineering in \textbf{Computer Science}; CGPA: \textbf{7.63}/10.00; 1\textsuperscript{st} Division with \textbf{Honours}}{2011 - 2015}
      
% What is it: High School at K.G. Children Hr. Sec. School, Gwalior by Board of Secondary Education, Madhya Pradesh in 2009
% What did I do: Passed in 1st Division with 88%. Distinction in all subjects
% What tech we used: NA
%------------------------------------------
  \resumeSubHeadingListEnd

\end{document}